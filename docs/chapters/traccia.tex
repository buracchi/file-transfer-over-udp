\section{Requisiti}

Lo scopo del progetto è quello di progettare ed implementare in linguaggi C
usando l'API del socket di Berkeley un'applicazione client-server per il
trasferimento di file che impieghi il servizio di rete senza connessione
(socket tipo SOCK\_DGRAM, ovvero UDP come protocollo di strato di trasporto.)
Il software deve permettere:

\begin{itemize}
    \item Connessione client-server senza autenticazione;
    \item La visualizzazione sul client dei file disponibili sul server
          (comando list);
    \item Il download di un file dal server (comando get);
    \item L'upload di un file sul server (comando put);
    \item Il trasferimento file in modo affidabile.
\end{itemize}

La comunicazione tra client e server deve avvenire tramite un opportuno
protocollo. Il protocollo di comunicazione deve prevedere lo scambio di due
tipi di messaggi:

\begin{enumerate}
    \item messaggi di comando: vengono inviati dal client al server per
          richiedere l'esecuzione delle diverse operazioni
    \item messaggi di risposta: vengono inviati dal server al client in risposta
          ad un comando con l'esito dell'operazione.
\end{enumerate}

\subsection{Funzionalità del server}

Il server, di tipo concorrente, deve fornire le seguenti funzionalità:

\begin{itemize}
    \item L’invio del messaggio di risposta al comando list al client
          richiedente; il messaggio di risposta contiene la filelist, ovvero la
          lista dei nomi dei file disponibili per la condivisione;
    \item L’invio del messaggio di risposta al comando get contenente il file
          richiesto, se presente, od un opportuno messaggio di errore;
    \item La ricezione di un messaggio put contenente il file da caricare sul
          server e l’invio di un messaggio di risposta con l’esito
          dell’operazione.
\end{itemize}

\subsection{Funzionalità del client}

I client, di tipo concorrente, devono fornire le seguenti funzionalità:

\begin{itemize}
    \item L’invio del messaggio list per richiedere la lista dei nomi dei file
          disponibili;
    \item L’invio del messaggio get per ottenere un file
    \item La ricezione di un file richiesta tramite il messaggio di get o la
          gestione dell’eventuale errore
    \item L’invio del messaggio put per effettuare l’upload di un file sul
          server e la ricezione del messaggio di risposta con l’esito
          dell’operazione.
\end{itemize}

\pagebreak
\subsection{Trasmissione affidabile}

Lo scambio di messaggi avviene usando un servizio di comunicazione non
affidabile. Al fine di garantire la corretta spedizione/ricezione dei messaggi
e dei file sia i client che il server implementano a livello applicativo il
protocollo Go-Back N
(cfr. Kurose \& Ross “Reti di Calcolatori e Internet”, 7° Edizione).

Per simulare la perdita dei messaggi in rete (evento alquanto improbabile in
una rete locale per non parlare di quando client e server sono eseguiti sullo
stesso host), si assume che ogni messaggio sia scartato dal mittente con
probabilità $p$.

La dimensione della finestra di spedizione $N$, la probabilità di perdita dei
messaggi $p$, e la durata del timeout $T$, sono tre costanti configurabili
ed uguali per tutti i processi. Oltre all'uso di un timeout fisso, deve essere
possibile scegliere l'uso di un valore per il timeout adattivo calcolato
dinamicamente in base alla evoluzione dei ritardi di rete osservati.

I client ed il server devono essere eseguiti nello spazio utente senza
richiedere privilegi di root. Il server deve essere in ascolto su una porta
di default (configurabile).
